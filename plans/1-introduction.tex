\section{Introduction}

Learning distributed representations (embeddings) of language has been a very
attractive topic in the recent development of natural language processing. Word
embedding assigns a (usually dense) low dimensional vector to each word which is
supposed to retain the semantic information. For example, the differences of
word vectors trained by Continuous bag-of-words or
Skipgram~\cite{mikolov2013distributed}, $\text{vec}(``woman'') -
\text{vec}(``man'') \approx \text{vec}(``king'') - \text{vec}(``queen'')$, are
found to be close.

Though it is advantageous to employ word embedding for language representation,
the effectiveness is inherently limited by its unigram assumption of language.
On one hand, the semantics of a word is context-unaware. For instance, the word
``bank'' in the sentences ``I made a deposit in the bank.'' and ``We walked on
the river bank.'' has the same embedding and hence semantics though a human
would interpret the word as different meanings based on the context. On the
other hand, it treats the semantics of higher level of language units (phrase,
sentence, and document) as an independent composition (linear function such as
averaging) of that of each constituent word. An unappealing implication is that
it oversimplifies the process of meaning formation of language. Under the
unigram assumption, ``the White House'' would has a greater similarity to ``a
house in white'' than ``presidential residence'', which is inconsistent with
human understanding.

Previous efforts towards resolving the semantics of phrase (and sentence) beyond
simple composition (averaging) have been made. Nevertheless by far most of them
are trained in a supervised manner where additional annotation of the text is
required. The annotation is necessary in order to provide richer information of
language structure than the sequence of words. Such additional annotation can be
a phrase dictionary~\cite{yin2014exploration},
POS-tagging~\cite{zhao2015phrase,baroni2010nouns} or syntactic structures from a
parser~\cite{levy2014dependency,yu2015learning,socher2013parsing,
le2015compositional,irsoy2014deep}. It is inspiring that by modeling the high
level language unit, the performance of text representation is greatly improved.
However, their development is achieved at the expense of requiring extra
annotation which can hardly be scalable for training on new corpora even new
languages.

In this work, we consider phrases as consecutive sequences of words in the text
and we present a simple yet effective algorithm to incorporate the notion of
phrase into the representation learning task in a fully unsupervised fashion.
This is essentially a much more challenging problem than previous investigated
settings since not only it learns the representation (embedding) but also it
infers the language structure from the unrestricted natural language text. Our
study, as a preliminary step to embark on the joint learning of language
structure and semantics, is based on the observation that \emph{phrasal
allocation} can be extracted from unstructured text by analyzing the occurrence
frequencies~\cite{witten1999kea,lindsey2012phrase,wang2007topical}, or
information-theoretic measurements such as pairwise mutual
information~\cite{fano1961transmission,church1990word} and generalized mutual
information~\cite{magerman1990parsing}. More importantly,
\cite{pantel2000unsupervised,collins1995prepositional} has demonstrated that
contextual similarity, or more precisely the semantic similarity of context, can
be leveraged to significantly boost the resolution of the prepositional phrasal
attachment. The strong mutual dependency between the two subtasks of structural
and semantic learning shows that potentially a great gain can be accomplished by
the joint modelling.

The proposed algorithm, \PLANS~(PLANS), jointly discoveries the phrase and
learns the embeddings. With a slight abuse of the notation, we call each single
word as well as a sequence of words as \emph{a phrase}. The first ingredient of
PLANS is that it models the allocation of phrase as a latent stochastic variable
generated from the \emph{transient Chinese restaurant process}~(tCRP).
Mathematically, for a word appeared in the context, it chooses its boundary of
the \emph{enclosing phrase}, with the probability specified similarly to that of
the Chinese restaurant process, encouraging the generation of frequent phrases.
Nevertheless, a computational challenge confronting PLANS is to retain only a
finite number of phrases while learning from a large corpus or stream dataset.
tCRP addresses this by letting customers periodically leave (with a probability)
the restaurant. In addition, tables in tCRP are sorted and pruned based on their
number of customers, which can be viewed as a generalization of the
\emph{frequency thresholding}. Another ingredient that underlies PLANS, namely
negative sampling~\cite{mikolov2013efficient} which is a popular technique
originally employed to train word embeddings, has its root in the estimation
method noise-contrastive estimation~\cite{gutmann2010noise} and is investigated
to train phrase embeddings. PLANS learns the phrases in an online manner and the
tables in tCRP are refreshed after the computation of each block of data, which
maintains a economic and reasonable set of phrases. The last ingredient of
PLANS, simulated annealing
(SA)~\cite{aarts1988simulated,brooks1995optimization}, is applied to the Gibbs
sampling in PLANS to stabilize the selected salient phrases while approaching
the end of learning. SA reduces the stochastic behavior of sampling over time as
the PLANS has more certainty about the language structure from training and
relies less on the sampling to explore the phrasal allocation.
